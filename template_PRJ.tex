\documentclass[a4paper,11pt]{book}
\usepackage[english]{babel}  
\usepackage[T1]{fontenc} 
\usepackage{helvet} % équivalent Arial

% R real
\usepackage{amsmath}
\usepackage{amssymb}
\newcommand{\R}{\mathbb{R}}

% L Laplace
\usepackage{mathrsfs}

% surlignage
\usepackage{soul}
\sethlcolor{yellow}

\usepackage[usenames]{xcolor}
\definecolor{darkpowderblue}{rgb}{0.0, 0.2, 0.6}
\definecolor{anti-flashwhite}{rgb}{0.95, 0.95, 0.96}
\definecolor{fireenginered}{rgb}{0.81, 0.09, 0.13}
\definecolor{ece}{RGB}{0, 122, 123}
\usepackage[colorlinks=true,linkcolor=black,urlcolor=darkpowderblue,citecolor=darkpowderblue]{hyperref}

\usepackage{listings} % needed for the inclusion of source code
\definecolor{codegreen}{rgb}{0,0.6,0}
\definecolor{codegray}{rgb}{0.5,0.5,0.5}
\definecolor{codepurple}{rgb}{0.58,0,0.82}
\definecolor{backcolour}{rgb}{0.95,0.95,0.92}
\definecolor{ao(english)}{rgb}{0.0, 0.5, 0.0}

\lstdefinestyle{mystyle}{
	backgroundcolor=\color{backcolour},   
	commentstyle=\color{ao(english)},
	keywordstyle=\color{blue},
	numberstyle=\tiny\color{codegray},
	stringstyle=\color{codepurple},
	basicstyle=\footnotesize,
	breakatwhitespace=false,         
	breaklines=true,                 
	captionpos=b,                    
	keepspaces=true,                 
	numbers=left,                    
	numbersep=5pt,                  
	showspaces=false,                
	showstringspaces=false,
	showtabs=false,                  
	tabsize=2
}

\lstdefinelanguage
[x64]{Assembler}     % add a "x64" dialect of Assembler
[x86masm]{Assembler} % based on the "x86masm" dialect
% with these extra keywords:
{morekeywords={CDQE,CQO,CMPSQ,CMPXCHG16B,JRCXZ,LODSQ,MOVSXD, %
		POPFQ,PUSHFQ,SCASQ,STOSQ,IRETQ,RDTSCP,SWAPGS, %
		rax,rdx,rcx,rbx,rsi,rdi,rsp,rbp, %
		r8,r8d,r8w,r8b,r9,r9d,r9w,r9b, %
		r10,r10d,r10w,r10b,r11,r11d,r11w,r11b, %
		r12,r12d,r12w,r12b,r13,r13d,r13w,r13b, %
		r14,r14d,r14w,r14b,r15,r15d,r15w,r15b,bcf,goto,btfsc,btfss,clrf,decfsz,return,incf}} % etc.
\lstset{style=mystyle}
\lstset{language=[x64]Assembler}
\setcounter{chapter}{-1} %starts @Chapter0


% exos sur deux colonnes
%\usepackage[font=it]{caption}% http://ctan.org/pkg/caption
%\usepackage{multicol,lipsum,graphicx,float}
%\usepackage{listings}
%\usepackage{multicol}
%\usepackage{pdflscape}

\definecolor{darkpowderblue}{rgb}{0.0, 0.2, 0.6}
\definecolor{denim}{rgb}{0.08, 0.38, 0.74}
\definecolor{gamboge}{rgb}{0.89, 0.61, 0.06}
\usepackage[pagestyles]{titlesec}
\titleformat{\section}
{\normalfont\fontsize{18}{12}\bfseries\color{ece}}{\thesection}{1em}{}
\titleformat{\subsection}
{\normalfont\fontsize{16}{12}\bfseries\color{ece}}{\thesubsection}{1em}{}
\titleformat{\subsubsection}{\normalfont\fontsize{15}{15}\bfseries\color{gamboge}}{\thesubsection}{1em}{}


%symboles
\usepackage{bclogo}


%F10 key
\usepackage{tikz}
\usetikzlibrary{shadows}
\newcommand*\keystroke[1]{%
	\tikz[baseline=(key.base)]
	\node[%
	draw,
	fill=white,
	drop shadow={shadow xshift=0.25ex,shadow yshift=-0.25ex,fill=black,opacity=0.75},
	rectangle,
	rounded corners=2pt,
	inner sep=1pt,
	line width=0.5pt,
	font=\scriptsize\sffamily
	](key) {#1\strut}
	;
}


% code
\usepackage{verbatim}

\newcommand{\HRule}[1]{\rule{\linewidth}{#1}}

%subfigure
\usepackage{graphicx,subcaption}
\usepackage{ragged2e}

% next year command

\newcommand\NextYear{%
	\advance\year by 1 \the\year\advance\year by -1}


% "Figure" en gras
\usepackage[labelfont=bf]{caption}

%tableaux horizontaux
\usepackage{booktabs}

%auteurs page de garde
\usepackage[affil-it]{authblk}
\makeatletter
\def\@maketitle{%
	\newpage
	\null
	\vskip 2em%
	\begin{center}%
		\let \footnote \thanks
		{\Large\bfseries \@title \par}%
		\vskip 1.5em%
		{\normalsize
			\lineskip .5em%
			\begin{tabular}[t]{c}%
				\@author
			\end{tabular}\par}%
		\vskip 1em%
		{\normalsize \@date}%
	\end{center}%
	\par
	\vskip 1.5em}
\makeatother

%numérotation parties / chapitres (reset counter)
\usepackage{hyperref}
%\usepackage{chngcntr}


% angstrom
\usepackage{siunitx}

% L lagrangien
\usepackage{ amssymb }

%caption dans minipage
\usepackage{caption}


% graphes
\usepackage{tikz}
\usepackage{xlop}
\newcommand*{\prettyprint}[1]{\opcopy{#1}{a}\opunzero{a}\opprint{a}}
\usepackage{pgfplots}



%signe euro
\usepackage{eurosym}

%ajout pdf
\usepackage[final]{pdfpages}

% sommaire
\usepackage[tight]{shorttoc}

% adresse mail
\usepackage{hyperref}
\newcommand{\email}[1]{\href{mailto:#1}{\nolinkurl{#1}}}
\urlstyle{sf}

% images
\usepackage{graphicx}

% maths
\usepackage{amssymb}
\usepackage{amsmath}
\usepackage{esint}

% part sans nvlle page
\usepackage{titlesec}
\titleclass{\part}{top}
\titleformat{\part}[display]
{\normalfont\huge\bfseries}{\centering\partname\ \thepart}{20pt}{\Huge\centering}
\titlespacing*{\part}{0pt}{50pt}{40pt}


% sous-subsection
\setcounter{secnumdepth}{4}
\setcounter{tocdepth}{4}
\makeatletter
\newcounter {subsubsubsection}[subsubsection]
\renewcommand\thesubsubsubsection{\thesubsubsection .\@alph\c@subsubsubsection}
\newcommand\subsubsubsection{\@startsection{subsubsubsection}{4}{\z@}%
	{-3.25ex\@plus -1ex \@minus -.2ex}%
	{1.5ex \@plus .2ex}%
	{\normalfont\normalsize\bfseries}}
\renewcommand\paragraph{\@startsection{paragraph}{5}{\z@}%
	{3.25ex \@plus1ex \@minus.2ex}%
	{-1em}%
	{\normalfont\normalsize\bfseries}}
\renewcommand\subparagraph{\@startsection{subparagraph}{6}{\parindent}%
	{3.25ex \@plus1ex \@minus .2ex}%
	{-1em}%
	{\normalfont\normalsize\bfseries}}
\newcommand*\l@subsubsubsection{\@dottedtocline{4}{10.0em}{4.1em}}
\renewcommand*\l@paragraph{\@dottedtocline{5}{10em}{5em}}
\renewcommand*\l@subparagraph{\@dottedtocline{6}{12em}{6em}}
\newcommand*{\subsubsubsectionmark}[1]{}
\makeatother
\usepackage{hyperref}
\makeatletter
\def\toclevel@subsubsubsection{4}
\def\toclevel@paragraph{5}
\def\toclevel@subparagraph{6}
\makeatother

% numérotation sous sous section en lettre
\renewcommand{\thesubsubsection}{\alph{subsubsection}}


%marges
\usepackage[margin=2cm]{geometry}

% texte en couleur
\usepackage{color}


% en-tête de page
\usepackage{fancyhdr}
\usepackage{lastpage}
\lhead{	\includegraphics[width=0.15\linewidth]{ece.png}}
\cfoot{\thepage}
\pagestyle{fancy}

%fraction spéciale
\newcommand*\rfrac[2]{{}^{#1}\!/_{#2}}

%lignes horizontale tableau épais
\usepackage{makecell}

% images
\usepackage{graphicx}

% maths
\usepackage{amssymb}
\usepackage{amsmath}

%\fancyfoot[L]{M. Schneider}
\fancyfoot[R]{}
\fancyfoot[L]{ING\hl{[X]}}


%fancy chapter presentation
\usepackage[Lenny]{fncychap}
%\ChNameVar{\Large \textbf{Laboratory Session}}

%\ChNumVar{\fontsize{90}{92}\bfseries}
%\ChTitleVar{\Huge\textbf{\thechapter}}
\usepackage{ifthen}
\makeatletter
\renewcommand{\@chapapp}{TP n$^{\text{o}}$}
\makeatother
\ChNameVar{\fontsize{20}{22}\usefont{OT1}{phv}{m}{n}\selectfont}
\ChNumVar{\fontsize{60}{62}\usefont{OT1}{ptm}{m}{n}\selectfont}
\ChTitleVar{\Huge\bfseries\rm}
\ChRuleWidth{1pt}

\renewcommand{\headrulewidth}{0pt}


%symboles drapeaux
\usepackage{bclogo}


\begin{document}
	
	\begin{titlepage}
	\title{\vspace{-5em}
		\begin{figure}[htb]
			\begin{minipage}[t]{.45\textwidth}
				\centering
				\includegraphics[width=\textwidth]{ece.png}
			\end{minipage}
			\hfill
			\begin{minipage}[t]{.45\textwidth}
				\centering
				\flushright\vspace{-12mm}\Large{\textbf{ING\hl{[X]}}\\ Groupe \hl{[X]}}
			\end{minipage}  
		\end{figure}
		\normalsize \textsc{} \\ [1.2cm]
		\HRule{1.5pt} \\ [0.5cm]
		\LARGE \textbf{\Large{RAPPORT DE PROJET}\\ \vspace{5mm}
			\Huge{\textcolor{ece}{PROJET \hl{[X]} : \hl{[Nom du projet]}}}}
		\HRule{1.5pt} \\ 
		\normalsize  \vspace{1cm}
		\fcolorbox{white}{anti-flashwhite}{\begin{minipage}{16cm}\vspace{4cm}\textbf{RÉSUMÉ} – Quel est le contexte et la problématique du projet ? Quels sont les objectifs techniques ? Dans quel contexte faites-vous ce projet ? [maximum 20 lignes]	\vspace{4cm}\end{minipage}}}
	\date{André-Marie	AMPÈRE, Alessandro	VOLTA\\ Nous attestons que ce travail est original, qu’il est le fruit d’un travail commun au binôme et qu’il a été rédigé de manière autonome. \\ \textbf{\hl{Paris, le JJ/MM/AAAA}}}
	
	\author{}
\end{titlepage}
	

	
	
	\maketitle
\section*{Table des matières}
I. Objectifs ................................................................................................................................. p. XX\\
II. Glossaire ................................................................................................................................ p. XX\\
A. Termes ................................................................................................................................... p. XX\\
B. Acronymes ............................................................................................................................. p. XX\\
III. L'équipe ............................................................................................................................... p. XX\\
A. Présentation de l'équipe ........................................................................................................ p. XX\\
B. Organisation de l'équipe ........................................................................................................ p. XX\\
C. Diagramme de GANTT ......................................................................................................... p. XX\\
IV. Contexte et problématique ................................................................................................... p. XX\\
A. Contexte ................................................................................................................................ p. XX\\
B. Problématique ....................................................................................................................... p. XX\\
C. Spécifications techniques ....................................................................................................... p. XX\\
V. Conception ............................................................................................................................. p. XX\\
A. Architecture fonctionnelle ...................................................................................................... p. XX\\
B. Architecture matérielle ........................................................................................................... p. XX\\
C. Architecture logicielle ............................................................................................................ p. XX\\
VI. Développement ..................................................................................................................... p. XX\\
A. Module 1 ................................................................................................................................ p. XX\\
B. Module 2 ................................................................................................................................ p. XX\\
C. Module 3 ................................................................................................................................ p. XX\\
VII. Tests et validation ............................................................................................................... p. XX\\
A. Module 1 ................................................................................................................................ p. XX\\
B. Module 2 ................................................................................................................................. p. XX\\
C. Module 3 ................................................................................................................................. p. XX\\
VIII. Bilan ................................................................................................................................... p. XX\\
A. État d'avancement .................................................................................................................. p. XX\\
B. Pertinence de la solution technique ........................................................................................ p. XX\\
C. Bilan sur le travail d'équipe .................................................................................................... p. XX\\
IX. Sources .................................................................................................................................. p. XX\\
X. Annexes .................................................................................................................................. p. XX\\









\newpage
\section*{I. Objectifs}
Quel est l’objectif de ce document ?\\
\noindent Que va y trouver le lecteur ?

\newpage
\section*{II. Glossaire}
\subsection*{A. Termes}
Renseigner ici sous forme de tableau les principaux termes techniques et leurs définitions.

\subsection*{B. Acronymes}
Renseigner ici sous forme de tableau les principaux acronymes, leurs signification et leurs explication.

\newpage
\section*{III. L'équipe}
\subsection*{A. Présentation de l'équipe}
Qui sont les membres qui composent l’équipe ?\\
\noindent Quelles sont leurs compétences et qualités ?

\subsection*{B. Organisation de l'équipe}
Comment est organisée l’équipe ? Comment est réparti le travail ?

\subsection*{C. Diagramme de Gantt}
Comment est utilisé le temps alloué au projet ? 

\newpage
\section*{IV. Contexte et problématique}
\subsection*{A. Contexte}
Quel est le contexte économique et ou sociétal du projet ?\\
\noindent Comment est née l’invention / la technologie du projet, comment a-t-elle évolué ?


\subsection*{B. Problématique}
À quelle problématique répond le projet ?

\subsection*{C. Spécifications techniques}
Quelles sont les spécifications techniques du projet ?\\
\noindent \textbf{NB} : Certains projets d’électronique à l’ECE n’en ont pas.

\newpage
\section*{V. Conception}
\subsection*{A. Architecture fonctionnelle}
Quelle est l’architecture fonctionnelle du projet ? \\
\noindent \textbf{NB} : Les fonctionnalités sont des verbes à l’infinitif suivi de compléments.\\
\noindent À ce stade, aucun choix technique n’est fait.

\subsection*{B. Architecture matérielle}
Quel matériel est utilisé et pourquoi ?\\
\noindent Comment les différentes briques techniques sont connectées entre elles ?\\
\noindent\textbf{NB} : Cela peut-être une schématique de circuit électronique.


\subsection*{C. Architecture logicielle}
Comment fonctionne le programme\\
\noindent \textbf{NB} : Présenter un algorigramme de votre code si vous en avez-un.

\newpage
\section*{VI. Développement}
L’idée est de présenter ici comment ont été développés les différents blocs du projet. Cela peut rassembler des calculs théoriques, des choix techniques, etc. et surtout bien expliquer le concept clef derrière sa fabrication.
Le lecteur doit être capable de comprendre les enjeux techniques et de développer le module en question à l’aide de ces sous-sections.

\subsection*{A. Module 1}
\subsection*{B. Module 2}
\subsection*{C. Module 3}


\newpage
\section*{VII. Tests et validation}
Une section au moins aussi importante que celle sur le développement.

\vspace{2mm}
\noindent Il est question ici de montrer les performances techniques du système et de valider le développement module par module puis au global (intégration) en accord avec la partie IV.

\vspace{2mm}
\noindent Chaque résultat (bien souvent des courbes) doit être décrit comme suit : 
\begin{itemize}
	\item Ce qui a été fait ;
	\item Ce que l’on est censé obtenir et critère de réussite du test ;
	\item Ce que l’on obtient ;
	\item Conclusion : validation ou non du bon fonctionnement du module.
\end{itemize}

\subsection*{A. Module 1}
\subsection*{B. Module 2}
\subsection*{C. Module 3}

\newpage
\section*{VIII. Bilan}
\subsection*{A. État d'avancement}
Où en est le projet ? A-t-on atteint les objectifs ?\\
\noindent Quels modules restent à finaliser (ou à perfectionner pour être en accord avec les spécifications techniques) ?

\subsection*{B. Pertinence de la solution technique}
Quelles sont les limites techniques de la solution développée ? \\
\noindent Quelles sont les possibilités d’évolution ou de poursuite ?


\subsection*{C. Bilan sur le travail d'équipe}
Qu’avez-vous appris individuellement ? Quelles compétences vont pouvoir être mises en avant lors de votre prochaine recherche de stage ?
\\
\noindent Comment l’équipe aurait pu mieux s’organiser ? Proposer un plan d’action pour le prochain projet.



\newpage
\section*{IX. Sources}
Documents utilisés et sites internet consultés pour développer le projet.\\
\noindent \textbf{NB} : voir le document « comment rédiger un rapport » sur la page Moodle La Toolbox pour la syntaxe à utiliser pour vos citations.



\newpage
\section*{X. Annexes}
Documents annexes, éventuels codes (\textbf{\textcolor{red}{pas de code dans le rapport}}).











\end{document}